% !TeX spellcheck = en_US
\documentclass[10pt,letterpaper,oneside]{article}

% this file contains all commonly used packages for the Rattlesnake documentation
% new package, add comment what they are used for

% command packages
\usepackage{xifthen}     % if -then else command

% language and font improvements
\usepackage[english]{babel}	% Multilingual support - https://www.ctan.org/pkg/babel
\usepackage[utf8]{inputenc} % Accept different input encodings - https://www.ctan.org/pkg/inputenc
\usepackage[kerning,spacing,babel]{microtype} % Subliminal refinements towards typographical perfection - https://www.ctan.org/pkg/microtype
\usepackage{eepic}      % Extensions to epic - https://www.ctan.org/pkg/eepic
\usepackage{epic}       % Enhance LaTeX picture mode - https://www.ctan.org/pkg/epic
\usepackage[T1]{fontenc} % selecting font encodingsselecting font encodings - https://www.ctan.org/pkg/fontenc
\usepackage{lmodern}

% title packages
\usepackage{authblk}		% footnote style author/affiliation - https://www.ctan.org/pkg/authblk

% symbols
\usepackage{upgreek}		% Upright Greek letters - https://www.ctan.org/pkg/upgreek
\usepackage{cancel}			% \cancel cmd - https://www.ctan.org/pkg/cancel

% chemical and isotope packages
\usepackage{isotope}		% isotope command - https://www.ctan.org/pkg/isotope
\usepackage{mhchem}			% chemical formulae/equations - https://www.ctan.org/pkg/mhchem

% layout packages
\usepackage{xspace}			% smart space \xspace - https://www.ctan.org/pkg/xspace
\usepackage{icomma}			% smart math comma - https://www.ctan.org/pkg/icomma
\usepackage{chngpage}		% page layout change during document - https://www.ctan.org/pkg/chngpage
\usepackage[usenames,dvipsnames,svgnames,table]{xcolor}			% color definitions - https://www.ctan.org/pkg/xcolor
\usepackage{enumitem}   % layout for itemize and enumerate - http://www.ctan.org/pkg/enumitem

% floating packages
% needed by the breakalgo environment labelsep=quad
\usepackage[singlelinecheck=false, labelsep=quad]{caption}	% table captions - https://www.ctan.org/pkg/caption
\usepackage{subcaption}		% Support for sub-captions - https://www.ctan.org/pkg/subcaption
\usepackage{placeins}   	% Control float placement - https://www.ctan.org/pkg/placeins
\usepackage{float}      	% Improved interface for floating objects - https://www.ctan.org/pkg/float

% table packages
\usepackage{supertabular}	% multi-page tables package - https://www.ctan.org/pkg/supertabular
\usepackage{booktabs}		% table settings - https://www.ctan.org/pkg/booktabs
\usepackage{array}			% Extending the array and tabular environments - https://www.ctan.org/pkg/array
\usepackage{multirow}		% tabular cells spanning multiple rows - https://www.ctan.org/pkg/multirow
% \usepackage{tabls}      % Better vertical spacing in tables and arrays - https://www.ctan.org/pkg/tabls

% pictures, plots and listings
\usepackage{graphicx}		% include graphics - https://www.ctan.org/pkg/graphicx
\usepackage{wrapfig}		% wrapfigure enviroment - https://www.ctan.org/pkg/wrapfig
\usepackage{pgfplots}		% latex plots - https://www.ctan.org/pkg/pgfplots

% source code and algorithms
\usepackage{algorithmicx} % newer version of algorithmic - https://www.ctan.org/pkg/algorithmicx
\usepackage{listings} 	% source code listings - http://www.ctan.org/pkg/listings
\usepackage{verbatim}   % verbatim - https://www.ctan.org/pkg/verbatim

% misc packages
\usepackage{url}        % provide \url command for web addresses - https://www.ctan.org/pkg/url
\usepackage{import}			% Establish input relative to a directory - https://www.ctan.org/pkg/import
\usepackage{afterpage}  % Execute command after the next page break - https://www.ctan.org/pkg/afterpage
\usepackage{lscape}     % Place selected parts of a document in landscape - https://www.ctan.org/pkg/lscape
\usepackage{rotating}   % Rotation tools, including rotated full-page floats - https://www.ctan.org/pkg/rotating
\usepackage{chngcntr}   % Change the resetting of counters - http://ctan.org/pkg/chngcntr

% eps support
\usepackage{epsfig}     % include eps figures - https://www.ctan.org/pkg/epsfig
\usepackage{epsf}       % Simple macros for EPS inclusion - https://www.ctan.org/pkg/epsf


% reference packages
% hyperref must be loaded before amsmath to avoid problems with subequations and align
\usepackage{hyperref}		% Extensive support for hypertext - https://www.ctan.org/pkg/hyperref


% packages that must be loaded after hyperref - http://tex.stackexchange.com/questions/1863/which-packages-should-be-loaded-after-hyperref-instead-of-before
\usepackage{geometry}		% Flexible and complete interface to document dimensions - https://www.ctan.org/pkg/geometry
\usepackage{algorithm}  % algorithm enviroment

% ams math packages
\usepackage{amsmath}    % mathematical facilities - https://www.ctan.org/pkg/amsmath
\usepackage{amsfonts}   % fonts for math - https://www.ctan.org/pkg/amsfonts
\usepackage{amssymb}    %
\usepackage{amsthm}     % Typesetting theorems - https://www.ctan.org/pkg/amsthm

\usepackage{nameref}		% allows references with names instread of numbers \nameref - http://www.ctan.org/pkg/nameref
\usepackage[capitalize,nameinlink]{cleveref}   % smart references - http://ctan.org/pkg/cleveref

% Common style and macro defintions for CLASS documents
%=================================================================================================

%%% General Commands

%% structural elements
% An explanation for a function
\newcommand{\explain}[1]{\mbox{\hspace{2em} #1}}
% attemtion box
\newcommand{\attention}[1]{\mbox{\hspace{2em} #1}}
% comments at the sde of the paragraph
\newcommand{\sidenotes}[1]{\marginpar{ {\footnotesize #1} }}
% create empty double page, TODO do we need that?
\newcommand{\clearemptydoublepage}{\newpage{\pagestyle{empty}\cleardoublepage}}
%TODO description here
\newcommand{\apost}{\textit{a posteriori\xspace}}
%TODO description here
\newcommand{\Apost}{\textit{A posteriori}\xspace}

%%% Math
%% Paratnhis etc.
% ()
\newcommand{\parenthesis}[1]{{\left(#1 \right)}}
% []
\newcommand{\bracket}[1]{{\left[ #1 \right]}}
% {}
\newcommand{\bracet}[1]{{\left\{#1 \right\}}}
% <>
\newcommand{\angled}[1]{{\left\langle#1\right\rangle}}

%% common math symbols
% imaginary number
\newcommand{\img}{\ensuremath{\hat{\imath}}\xspace}
% gradient symbol
\newcommand{\grad}{\vec{\nabla}}
\newcommand{\del}{\vec{\nabla}}
% adjoint
\newcommand{\adj}[2][{}]{{{#2}^{\dagger#1}}}
% order number
\newcommand{\order}[1]{^\parenthesis{#1}}
% iteration number
\newcommand{\iter}[1]{^{#1}}

%% common math function
% divergence
\renewcommand{\div}{\del\! \cdot \!}
% rotation
\newcommand{\rot}{\del\! \times \!}
% absolute value
\newcommand{\abs}[1]{\left|#1\right|}
% norm
\newcommand{\norm}[2][{}]{\lVert#2\rVert_{#1}}
% e function
\newcommand{\e}[1]{\mathrm{e}^{#1}}
% power of ten
\newcommand{\tento}[1]{\ensuremath{10^{#1}}\xspace}
% E notation
\newcommand{\E}[1]{\ensuremath{\cdot \tento{#1}}\xspace}
% sign
\DeclareMathOperator{\sign}{sign}

%% fraction
% 1 / 2
\newcommand{\half}[1][1]{\frac{#1}{2}}
% 1 / 3
\newcommand{\third}[1][1]{\frac{#1}{3}}
% 1 / 4
\newcommand{\fourth}[1][1]{\frac{#1}{4}}

% vectors etc
% vector, we use standard latex notation
% matrix
\newcommand{\mat}[1]{\mathbf{#1}}
% tensor
\newcommand{\tensor}[1]{\underline{\underline{#1}}}
% operator symbol
\newcommand{\op}[1]{\mathrm{\mathbf{#1}}}

% derivatives
% first derivatives
% general first derivative, with optional argument for f
\newcommand{\dd}[2][]{\frac{\partial #1}{\partial #2}}
% partial derivative for x, optional argument for f
\newcommand{\ddx}[1][]{\dd[#1]{x}}
% partial derivative for y, optional argument for f
\newcommand{\ddy}[1][]{\dd[#1]{y}}
% partial derivative for z, optional argument for f
\newcommand{\ddz}[1][]{\dd[#1]{z}}
% partial derivative for t, optional argument for f
\newcommand{\ddt}[1][]{\dd[#1]{t}}

% second derivatives, only for a single variable, cross derivatives are too many
% second general derivative, optional argument for f
\newcommand{\ddd}[2][{}]{\frac{\partial^2 #1}{\partial {#2}^2}}
% second partial derivative for x, optional argument for f
\newcommand{\ddxx}[1][]{\ddd[#1]{x}}
% second partial derivative for y, optional argument for f
\newcommand{\ddyy}[1][]{\ddd[#1]{y}}
% second partial derivative for z, optional argument for f
\newcommand{\ddzz}[1][]{\ddd[#1]{z}}
% second partial derivative for t, optional argument for f
\newcommand{\ddtt}[1][]{\ddd[#1]{t}}

% integrals
% integral dx, optional parameter for different variable
\newcommand{\dx}[1][x]{\,d#1}
% integral dy
\newcommand{\dy}{\dx[y]}
% integral dz
\newcommand{\dz}{\dx[z]}
% integral dt
\newcommand{\dt}{\dx[t]}
%integral dmu
\newcommand{\dmu}{\dx[\mu]}
%integral dOmega
\newcommand{\domg}{\dx[\direction]}

% spherical integrals
% shortcut for sphere notation
\newcommand{\sphere}{\ensuremath{\mathcal{S}}\xspace}
% angular quadrature weight
\newcommand{\aqweight}{\omega}

% integral over full sphere
\newcommand{\intsp}{\int_{4\pi}}
% integral over half sphere
\newcommand{\inthalfsp}{\int_{2\pi}}
% polar integral
\newcommand{\intpolar}{\int_{-1}^{1}}
% negative partial polar integral
\newcommand{\intnpolar}{\int_{-1}^{0}}
% positive partial polar integral
\newcommand{\intppolar}{\int_{0}^{1}}


% FEM symbols
% spatial domain
\newcommand*{\domain}{\ensuremath{\mathcal{D}}\xspace}
% boundary of a spatical domain
\newcommand*{\boundary}{\ensuremath{{\partial\domain}}\xspace}
% vacuum boundary of a spatical domain
\newcommand*{\vboundary}{\ensuremath{{\boundary}_v}\xspace}
% reflective boundary of a spatical domain
\newcommand*{\rboundary}{\ensuremath{{\boundary}_r}\xspace}
% interface surface
\newcommand*{\interface}{\ensuremath{{\Gamma}}\xspace}
% general and isotropic test function
\newcommand{\testfct}{\ensuremath{\phi^{*}}\xspace}
% angular test function
\newcommand{\atestfct}{\ensuremath{\psi^{*}}\xspace}
% surface normal
\newcommand{\normal}{\ensuremath{\vec{n}}\xspace}
% boundary normal
\newcommand{\bnormal}{\ensuremath{\normal_\mathrm{b}}\xspace}

% DFEM commands
% interface jump
\newcommand{\jump}[1]{[\![#1]\!]}
% what is the different?
\newcommand{\jmpa}[1]{[\![\![#1]\!]\!]}
% mean value
\newcommand{\meanval}[1]{\{\!\!\{#1\}\!\!\}}

% common physical symbols
% mass stream
\newcommand{\mdot}{\ensuremath{\dot{m}}\xspace}

% nuclear symbols
% Sn
\newcommand{\sn}[1][N]{\ensuremath{S_#1}\xspace}
% Pn
\newcommand{\pn}[1][N]{\ensuremath{P_#1}\xspace}
% keff
\newcommand{\keff}{\ensuremath{k_{\text{eff}}}\xspace}
% kinf
\newcommand{\kinf}{\ensuremath{k_{\text{inf}}}\xspace}

% transport symbols
\newcommand{\addgroup}[1]{\ifthenelse{\isempty{#1}}{}{_{#1}}}
% direction omega
\newcommand{\direction}{\ensuremath{\vec{\Omega}}\xspace}
% direction omega
\newcommand{\position}{\ensuremath{\vec{x}}\xspace}
% current
\newcommand{\current}[1][]{\ensuremath{\vec{J}\addgroup{#1}}\xspace}
% positive half range current
\newcommand{\ppcurrent}[1][]{\ensuremath{\hat{\jmath}\,^+\addgroup{#1}}\xspace}
% negative half range current
\newcommand{\npcurrent}[1][]{\ensuremath{\hat{\jmath}\,^-\addgroup{#1}}\xspace}
% drift vector
\newcommand{\drift}[1][]{\ensuremath{\hat{D}\addgroup{#1}}\xspace}
% local diffusion coefficient
\newcommand{\DC}[1][]{\ensuremath{\mathrm{D}\addgroup{#1}}\xspace}
% nonlocal diffusion tensor
\newcommand{\DCNL}[1][]{\ensuremath{\tensor{\mathrm{D}}\addgroup{#1}}\xspace}
% cross section label
\newcommand{\xslabel}[2][]{\ifthenelse{\isempty{#1}}{\mathrm{#2}}{\mathrm{#2},#1}}
% total cross section
\newcommand{\sigt}[1][]{\ensuremath{\sigma_{\xslabel[#1]{t}}}\xspace}
% scattering cross section
\newcommand{\sigs}[1][]{\ensuremath{\sigma_{\xslabel[#1]{s}}}\xspace}
% fission cross section
\newcommand{\sigf}[1][]{\ensuremath{\sigma_{\xslabel[#1]{f}}}\xspace}
% removal cross section
\newcommand{\sigr}[1][]{\ensuremath{\sigma_{\xslabel[#1]{r}}}\xspace}
% absorption cross section
\newcommand{\siga}[1][]{\ensuremath{\sigma_{\xslabel[#1]{a}}}\xspace}
% transport cross section
\newcommand{\sigtr}[1][]{\ensuremath{\sigma_{\xslabel[#1]{tr}}}\xspace}
% scattering moment
\newcommand{\sigl}[2][{}]{\ensuremath{\sigma_{#2#1}}\xspace}

% total cross section
\newcommand{\Sigt}[1][]{\ensuremath{\Sigma_{\xslabel[#1]{t}}}\xspace}
% scattering cross section
\newcommand{\Sigs}[1][]{\ensuremath{\Sigma_{\xslabel[#1]{s}}}\xspace}
% fission cross section
\newcommand{\Sigf}[1][]{\ensuremath{\Sigma_{\xslabel[#1]{f}}}\xspace}
% removal cross section
\newcommand{\Sigr}[1][]{\ensuremath{\Sigma_{\xslabel[#1]{r}}}\xspace}
% absorption cross section
\newcommand{\Siga}[1][]{\ensuremath{\Sigma_{\xslabel[#1]{a}}}\xspace}
% transport cross section
\newcommand{\Sigtr}[1][]{\ensuremath{\Sigma_{\xslabel[#1]{tr}}}\xspace}
% scattering moment
\newcommand{\Sigl}[2][{}]{\ensuremath{\Sigma_{#2#1}}\xspace}

% spatial weight function
\newcommand{\weight}[1][]{\ensuremath{w\addgroup{#1}}\xspace}

% physical units
% general style for a unit symbol
\newcommand{\unit}[1]{\mathrm{#1}}
% distance
% meter maybe \meter better and more clear?
\newcommand{\m}{\,\unit{m}\xspace}
% centimeter
\newcommand{\cm}{\,\unit{cm}\xspace}
% milimeter
\newcommand{\mm}{\,\unit{mm}\xspace}

% time
% second
\newcommand{\s}{\,\unit{s}\xspace}

% transport units
% scalar flux
\newcommand{\sfluxunit}{\,\ensuremath{\frac{1}{\unit{cm}^2\unit{s}}}\xspace}
% angular flux
\newcommand{\afluxunit}{\,\ensuremath{\frac{1}{\unit{cm}^2\unit{s}\cdot\unit{st}}}\xspace}
% diffusion coefficient
\newcommand{\dcunit}{\,\ensuremath{\unit{cm}}\xspace}


% \newenvironment{myverbatim}%            To change the pseudocode font
% {\par\noindent%
%  \rule[0pt]{\linewidth}{0.2pt}
%  \vspace*{-9pt}
%  \linespread{0.0}\small\verbatim}%
% {\rule[-5pt]{\linewidth}{0.2pt}\endverbatim}
%
% \newenvironment{myverbatim1}%            To change the pseudocode font
% {\par\noindent%
%  \rule[0pt]{\linewidth}{0.2pt}
%  \vspace*{-9pt}
%  \linespread{1.0}\scriptsize\verbatim}%
% {\rule[-5pt]{\linewidth}{0.2pt}\endverbatim}

% general settings
\renewcommand\floatpagefraction{0.75} % threshold for floating objects to be placed on a page

\makeatletter
% center floats generally
\g@addto@macro\@floatboxreset\centering
% gives the current section name
\newcommand*{\currentname}{\@currentlabelname}
\makeatother

% figure setting, controlles the height of pgfplots from the main script
\newlength{\figheight}
\setlength{\figheight}{12cm}

% example for hyperref setup, we have to see if we can get that better
\hypersetup{
  linktocpage=true,       % links in table of contents
  bookmarks=true,         % show bookmarks bar?
  unicode=true,           % non-Latin characters in Acrobat's bookmarks
  pdftoolbar=true,        % show Acrobat's toolbar?
  pdfmenubar=true,        % show Acrobat's menu?
  pdffitwindow=false,     % window fit to page when opened
  pdfstartview={FitH},    % fits the width of the page to the window
  pdfnewwindow=true,      % links in new window
  colorlinks=false,       % false: boxed links; true: colored links
  linkcolor=red,          % color of internal links
  citecolor=green,        % color of links to bibliography
  filecolor=magenta,      % color of file links
  urlcolor=cyan,          % color of external links
  pdfborder={0 0 0},      % color of link frames
}


% set package versions
\pgfplotsset{compat=1.12}

% set table and figure counters here

% set up listings - see https://en.wikibooks.org/wiki/LaTeX/Source_Code_Listings
\lstset{ %
  backgroundcolor=\color{white},   % choose the background color; you must add \usepackage{color} or \usepackage{xcolor}; should come as last argument
  basicstyle=\footnotesize,        % the size of the fonts that are used for the code
  breakatwhitespace=false,         % sets if automatic breaks should only happen at whitespace
  breaklines=true,                 % sets automatic line breaking
  captionpos=b,                    % sets the caption-position to bottom
  commentstyle=\color{Gray},       % comment style
  deletekeywords={...},            % if you want to delete keywords from the given language
  escapeinside={\%*}{*)},          % if you want to add LaTeX within your code
  extendedchars=true,              % lets you use non-ASCII characters; for 8-bits encodings only, does not work with UTF-8
  frame=single,	                   % adds a frame around the code
  keepspaces=true,                 % keeps spaces in text, useful for keeping indentation of code (possibly needs columns=flexible)
  keywordstyle=\color{blue},       % keyword style
  morekeywords={*,...},            % if you want to add more keywords to the set
  numbers=left,                    % where to put the line-numbers; possible values are (none, left, right)
  numbersep=5pt,                   % how far the line-numbers are from the code
  numberstyle=\tiny\color{Gray},   % the style that is used for the line-numbers
  rulecolor=\color{black},         % if not set, the frame-color may be changed on line-breaks within not-black text (e.g. comments (green here))
  showspaces=false,                % show spaces everywhere adding particular underscores; it overrides 'showstringspaces'
  showstringspaces=false,          % underline spaces within strings only
  showtabs=false,                  % show tabs within strings adding particular underscores
  stepnumber=2,                    % the step between two line-numbers. If it's 1, each line will be numbered
  stringstyle=\color{mymauve},     % string literal style
  tabsize=2,	                     % sets default tabsize to 2 spaces
  title=\lstname                   % show the filename of files included with \lstinputlisting; also try caption instead of title
}

% define style for moose
\lstdefinestyle{cpp}{
  belowcaptionskip=1\baselineskip,
  breaklines=true,
  frame=TB,
  xleftmargin=\parindent,
  language=C,
  showstringspaces=false,
  basicstyle=\footnotesize\ttfamily,
  keywordstyle=\bfseries\color{green!40!black},
  commentstyle=\itshape\color{purple!40!black},
  identifierstyle=\color{blue},
  stringstyle=\color{orange},
}

\lstdefinestyle{python}{
  belowcaptionskip=1\baselineskip,
  breaklines=true,
  frame=,
  xleftmargin=\parindent,
  language=Python,
  showstringspaces=false,
  basicstyle=\footnotesize\ttfamily,
  keywordstyle=\bfseries\color{blue},
  commentstyle=\itshape\color{gray},
  identifierstyle=\color{black},
  stringstyle=\color{green},
}

\lstdefinelanguage{Moose}{
  morekeywords={one,two,three,four,five,six,seven,eight,
  nine,ten,eleven,twelve,o,clock,rock,around,the,tonight},
  sensitive=false,
  morecomment=[l]{//},
  morecomment=[s]{/*}{*/},
  morestring=[b]",
}

\lstdefinestyle{moose}{
  belowcaptionskip=1\baselineskip,
  breaklines=true,
  frame=tb,
  xleftmargin=\parindent,
  language=Moose,
  showstringspaces=false,
  basicstyle=\footnotesize\ttfamily,
  keywordstyle=\bfseries\color{blue},
  commentstyle=\itshape\color{gray},
  identifierstyle=\color{black},
  stringstyle=\color{green},
}

\lstset{literate=
  {á}{{\'a}}1 {é}{{\'e}}1 {í}{{\'i}}1 {ó}{{\'o}}1 {ú}{{\'u}}1
  {Á}{{\'A}}1 {É}{{\'E}}1 {Í}{{\'I}}1 {Ó}{{\'O}}1 {Ú}{{\'U}}1
  {à}{{\`a}}1 {è}{{\`e}}1 {ì}{{\`i}}1 {ò}{{\`o}}1 {ù}{{\`u}}1
  {À}{{\`A}}1 {È}{{\'E}}1 {Ì}{{\`I}}1 {Ò}{{\`O}}1 {Ù}{{\`U}}1
  {ä}{{\"a}}1 {ë}{{\"e}}1 {ï}{{\"i}}1 {ö}{{\"o}}1 {ü}{{\"u}}1
  {Ä}{{\"A}}1 {Ë}{{\"E}}1 {Ï}{{\"I}}1 {Ö}{{\"O}}1 {Ü}{{\"U}}1
  {â}{{\^a}}1 {ê}{{\^e}}1 {î}{{\^i}}1 {ô}{{\^o}}1 {û}{{\^u}}1
  {Â}{{\^A}}1 {Ê}{{\^E}}1 {Î}{{\^I}}1 {Ô}{{\^O}}1 {Û}{{\^U}}1
  {œ}{{\oe}}1 {Œ}{{\OE}}1 {æ}{{\ae}}1 {Æ}{{\AE}}1 {ß}{{\ss}}1
  {ű}{{\H{u}}}1 {Ű}{{\H{U}}}1 {ő}{{\H{o}}}1 {Ő}{{\H{O}}}1
  {ç}{{\c c}}1 {Ç}{{\c C}}1 {ø}{{\o}}1 {å}{{\r a}}1 {Å}{{\r A}}1
  {€}{{\euro}}1 {£}{{\pounds}}1
}


% example for hyperref setup, these are the fields that need to be updated
\hypersetup{
  pdftitle={Test file for Latex definitions},    % title
  pdfauthor={Hans Hammer},     % author
  pdfsubject={Documentation},   % subject of the document
  pdfcreator={Hans Hammer},   % creator of the document
  pdfproducer={Hans Hammer}, % producer of the document
  pdfkeywords={Documentation}, % list of keywords
}

% define title and all authors here
\title{Documentation test document}
\author[1]{Hans Hammer}
\author[1]{Tarek Ghaddar}

\affil[1]{Texas A\&M University - Department of Nuclear Engineering}

\begin{document}
\maketitle

\tableofcontents

\section{Introduction}
This documents shows the usage of the general style definitions for documentation. Most command are defined using the math mode. Even some are directly usable in free text, they still switch to math mode. This changes the font and style. Keep that in mind.
If the math mode must be enabled manually, this is shown in the examples.

\subsection{References}
Use the cref package for references. It will detect automatically the type of reference and add the according description:

\begin{figure}[H]
  \caption{Empty Figure with label \texttt{fig:label\_1}}
  \label{fig:label_1}
\end{figure}

\begin{table}[H]
  \caption{Empty Table with label \texttt{tab:label\_2}}
  \label{tab:label_2}
\end{table}

\begin{equation} \label{eq:label_3}
  1 + 1 = 2
\end{equation}

Reference with \lstinline|\cref{fig:label_1}| for \cref{fig:label_1} and with  \lstinline|\cref{tab:label_2}| for \cref{tab:label_2}. For the beginning of a sentence use \lstinline|\Cref{fig:label_1}| for \Cref{fig:label_1} and \lstinline|\Cref{tab:label_2}| for \Cref{tab:label_2}. Reference texts can be adjusted, see package documentation for details.
Same for equations \lstinline|\cref{eq:label_3}| for \cref{eq:label_3} and \lstinline|\Cref{eq:label_3}| for \Cref{eq:label_3} for begining of the sentence.
Fancy is also \lstinline|\cref{fig:label_1, tab:label_2, eq:label_3}| for \cref{fig:label_1,tab:label_2,eq:label_3}.
The prefix before the colon is not needed and just added for human readablility. These are shwon in \cref{tab:label_prefixes}. A label shoud be as descriptive as possible.
\begin{table}
  \caption{Prefixes for reference labels}
  \begin{tabular}{l|c}
    Type & prefix \\ \hline
    Equation & eq: \\
    Table & tab: \\
    Figure & fig: \\
    Algorithm & alg: \\
    Listing & lst: \\
  \end{tabular}
  \label{tab:label_prefixes}
\end{table}

\subsection{Indices}
Where do group, angle, iteration and spatial indices go? Which symbols shall we use?


\subsection{Listings}
Look into source code to see how it is done. You can also load code from files directly, even just some  lines from the file. See documentation.
\url{https://en.wikibooks.org/wiki/LaTeX/Source_Code_Listings}. I am still working on the styles, this are just examples at the moment
\begin{lstlisting}[style=python,caption={Python example},label={lst:python}]
import numpy as np

def incmatrix(genl1,genl2):
    m = len(genl1)
    n = len(genl2)
    M = None #to become the incidence matrix
    VT = np.zeros((n*m,1), int)  #dummy variable

    #compute the bitwise xor matrix
    M1 = bitxormatrix(genl1)
    M2 = np.triu(bitxormatrix(genl2),1)

    for i in range(m-1):
        for j in range(i+1, m):
            [r,c] = np.where(M2 == M1[i,j])
            for k in range(len(r)):
                VT[(i)*n + r[k]] = 1;
                VT[(i)*n + c[k]] = 1;
                VT[(j)*n + r[k]] = 1;
                VT[(j)*n + c[k]] = 1;

                if M is None:
                    M = np.copy(VT)
                else:
                    M = np.concatenate((M, VT), 1)

                VT = np.zeros((n*m,1), int)

    return M
\end{lstlisting}

\begin{lstlisting}[style=cpp,caption={C++ example},label={lst:moose}]
#include <stdio.h>
#define N 10
/* Block
 * comment */

int main()
{
    int i;

    // Line comment.
    puts("Hello world!");

    for (i = 0; i < N; i++)
    {
        puts("LaTeX is also great for programmers!");
    }

    return 0;
}
\end{lstlisting}

\begin{lstlisting}[style=moose,caption={Moose input example},label={lst:moose}]
[Mesh]
  type = GeneratedBIDMesh
  dim = 1
  xmin = 0
  xmax = 2
  nx = 2
  subdomain = '0 1'
[]

[TransportSystems]
  particle = common
  equation_type = steady-state
  G = 1
  VacuumBoundary = 'left right'
  SurfaceSource = '10.0 0'

  [./diffusion]
    scheme = CFEM-Diffusion
    order = FIRST
    family = LAGRANGE
    nonlocal_diffusion_multiapp_file = 'absorber_larsen_trahan_ls.i'
    save_nonlocal_diffusion_coefficient = true

    transport_wrapper = 'ls_transport'
  [../]
[]

[Materials]
  [./strong]
    type = ConstantNeutronicsMaterial
    block = '1'
    sigma_t = 10.0
    sigma_s = 5.0
  [../]
  [./weak]
    type = ConstantNeutronicsMaterial
    block = '0'
    sigma_t = 0.1
    sigma_s = 0.05
  [../]
[]
}
\end{lstlisting}

\subsection{Useful links}
\subsubsection{Spaces in \LaTeX}
\begin{table}[H]
  \caption{Spacing from \url{http://tex.stackexchange.com/questions/74353/what-commands-are-there-for-horizontal-spacing}}
  \label{tab:spacing}
  \noindent\begin{supertabular}{lp{5cm}}
    \verb|a\,b| & a\,b \\
    \verb|$a\,b$| & $a\,b$ \\
    \verb|a\thinspace b| & a\thinspace b \\
    \verb|$a\thinspace b$| & $a\thinspace b$ \\
    \verb|$a\!b$| & $a\!b$ \\
    \verb|$a\mkern-\thinmuskip b$| & $a\mkern-\thinmuskip b$ \\
    \verb|$a\>b$| & $a\>b$ \\
    \verb|$a\mkern\medmuskip b$| & $a\mkern\medmuskip b$ \\
    \verb|$a\;b$| & $a\;b$ \\
    \verb|$a\mkern\thickmuskip b$| & $a\mkern\thickmuskip b$ \\
    \verb|$a\:b$| & $a\:b$ \\
    \verb|$a\mkern\medmuskip b$| & $a\mkern\medmuskip b$ \\
    \verb|a\enspace b| & a\enspace b \\
    \verb|$a\enspace b$| & $a\enspace b$ \\
    \verb|a\quad b| & a\quad b \\
    \verb|$a\quad b$| & $a\quad b$ \\
    \verb|a\qquad b| & a\qquad b \\
    \verb|$a\qquad b$| & $a\qquad b$ \\
    \verb|a\hskip 1em b| & a\hskip 1em b \\
    \verb|$a\hskip 1em b$| & $a\hskip 1em b$ \\
    \verb|a\kern 1pc b| & a\kern 1pc b \\
    \verb|$a\kern 1pc b$| & $a\kern 1pc b$ \\
    \verb|a\hspace{35pt}b| & a\hspace{35pt}b \\
    \verb|$a\hspace{35pt}b$| & $a\hspace{35pt}b$ \\
    \verb|axyzb| & axyzb \\
    \verb|a\hphantom{xyz}b| & a\hphantom{xyz}b \\
    \verb|$axyzb$| & $axyzb$ \\
    \verb|$a\hphantom{xyz}b$| & $a\hphantom{xyz}b$ \\
    \verb|a\ b| & a\ b \\
    \verb|$a\ b$| & $a\ b$ \\
    \verb|a~b| & a~b \\
    \verb|$a~b$| & $a~b$ \\
    \verb|a\hfill b| & a\hfill b \\
    \verb|$a\hfill b$| & $a\hfill b$ \\
  \end{supertabular}
\end{table}

\subsection{AMSMath package}
Read the documentation on the ams math environments \url{ftp://ftp.ams.org/pub/tex/doc/amsmath/amsldoc.pdf}

\subsection{Plotting}
\LaTeX supports plots from csv files. The learning curve is a bit steep, however the results are worth it. Take a look at the pgfplots package \url{ftp://ftp.ams.org/pub/tex/doc/amsmath/amsldoc.pdf}. I might add a 2\,D template for plots later.

% give us some more space
\renewcommand{\arraystretch}{2}

\section{General commands}
\begin{supertabular}{p{0.4\textwidth}|p{0.3\textwidth}|p{0.3\textwidth}}
  Name & Symbol & Command \\ \hline
\end{supertabular}

\section{Math}
\subsection{Parenthesis \& Co}
\begin{supertabular}{p{0.4\textwidth}|p{0.3\textwidth}|p{0.3\textwidth}}
  Name & Symbol & Command \\ \hline
  Parenthesis  & $\parenthesis{x\cdot x}$  &  \lstinline|$\parenthesis{x\cdot x}$|\\
  Bracket  & $\bracket{x\cdot x}$  &  \lstinline|$\bracket{x\cdot x}$|\\
  Bracet & $\bracet{x\cdot x}$  &  \lstinline|$\bracet{x\cdot x}$|\\
  Angled &  $\angled{x\cdot x}$  &  \lstinline|$\angled{x\cdot x}$|\\
\end{supertabular}

\subsection{Math symbols}
\begin{supertabular}{p{0.4\textwidth}|p{0.3\textwidth}|p{0.3\textwidth}}
  Name & Symbol & Command \\ \hline
  Imaginary number &  \img  &  \lstinline|\img|\\
  Gradient &  $\del$  &  \lstinline|$\grad$| or \lstinline|$\del$|\\
  Adjoint &  $\adj{\psi}$  &  \lstinline|$\adj{\psi}$|\\
  Order &  $e\order{2}$  &  \lstinline|$e\order{2}$|\\
\end{supertabular}

\subsection{Functions}
\begin{supertabular}{p{0.4\textwidth}|p{0.3\textwidth}|p{0.3\textwidth}}
  Name & Symbol & Command \\ \hline
  Divergence &  $\div\psi$  &  \lstinline|$\div\psi$|\\
  Rotation &  $\rot\psi$  &  \lstinline|$\rot\psi$|\\
  Vector Norm &  $\norm{\vec{x}}$  &  \lstinline|$\norm{\vec{x}}$| \\
   & $\norm[\infty]{\vec{x}}$  &  \lstinline|$\norm[\infty]{\vec{x}}$|\\
  Absolute value &  $\abs{x}$  &  \lstinline|$\abs{x}$|\\
  e-Function &  $\e{x}$  &  \lstinline|$\e{x}$|\\
  Power of ten &  \tento{3}  &  \lstinline|\tento{3}|\\
  Scientific noation &  8.3\E{-4}  &  \lstinline|8.3\E{-4}|\\
  Sign function &  $\sign$  &  \lstinline|$\sign$|\\
\end{supertabular}

\subsection{Matrices etc}
\begin{supertabular}{p{0.4\textwidth}|p{0.3\textwidth}|p{0.3\textwidth}}
  Name & Symbol & Command \\ \hline
  Vector &  $\vec{A}$  &  \lstinline|$\vec{A}$|\\
  Matrix &  $\mat{A}$  &  \lstinline|$\mat{A}$|\\
  Tensor &  $\tensor{T}$  &  \lstinline|$\tensor{T}$|\\
  Function Operator &  $\op{L}$  &  \lstinline|$\op{L}$|\\
\end{supertabular}

\subsection{Derivatives}
\subsubsection{First Derivatives}
\begin{supertabular}{p{0.4\textwidth}|p{0.3\textwidth}|p{0.3\textwidth}}
  Name & Symbol & Command \\ \hline
  General derivative &  $\dd{s}$  &  \lstinline|$\dd{s}$| \\
   & $\dd[f]{s}$  &  \lstinline|$\dd[f]{s}$|\\
  x derivative &  $\ddx$  &  \lstinline|$\ddx$| \\
   & $\ddx[f]$  &  \lstinline|$\ddx[f]$|\\
  y derivative &  $\ddy$  &  \lstinline|$\ddy$| \\
   & $\ddy[f]$  &  \lstinline|$\ddy[f]$|\\
  z derivative &  $\ddz$  &  \lstinline|$\ddz$| \\
   & $\ddz[f]$  &  \lstinline|$\ddz[f]$|\\
  t derivative &  $\ddt$  &  \lstinline|$\ddt$| \\
   & $\ddt[f]$  &  \lstinline|$\ddt[f]$|\\
\end{supertabular}

\subsection{Second Derivatives}
\begin{supertabular}{p{0.4\textwidth}|p{0.3\textwidth}|p{0.3\textwidth}}
  Name & Symbol & Command \\ \hline
  General derivative &  $\ddd{s}$  &  \lstinline|$\ddd{s}$| \\
   & $\ddd[f]{s}$  &  \lstinline|$\ddd[f]{s}$|\\
  x derivative &  $\ddxx$  &  \lstinline|$\ddxx$| \\
   & $\ddxx[f]$  &  \lstinline|$\ddxx[f]$|\\
  y derivative &  $\ddyy$  &  \lstinline|$\ddyy$| \\
   & $\ddyy[f]$  &  \lstinline|$\ddyy[f]$|\\
  z derivative &  $\ddzz$  &  \lstinline|$\ddzz$| \\
   & $\ddzz[f]$  &  \lstinline|$\ddzz[f]$|\\
  t derivative &  $\ddtt$  &  \lstinline|$\ddtt$| \\
   & $\ddtt[f]$  &  \lstinline|$\ddtt[f]$|\\
\end{supertabular}

\subsection{Integral}
\begin{supertabular}{p{0.4\textwidth}|p{0.3\textwidth}|p{0.3\textwidth}}
  Name & Symbol & Command \\ \hline
  x Integrate &  $\dx$  &  \lstinline|$\dx$| \\
   & $\dx[s]$  &  \lstinline|$\dx[s]$|\\
  y Integrate &  $\dy$  &  \lstinline|$\dy$|\\
  z Integrate &  $\dz$  &  \lstinline|$\dz$|\\
  t Integrate &  $\dt$  &  \lstinline|$\dt$|\\
\end{supertabular}

\subsubsection{Sperhical Integral}
\begin{supertabular}{p{0.4\textwidth}|p{0.3\textwidth}|p{0.3\textwidth}}
  Name & Symbol & Command \\ \hline
  Sphere & \sphere & \lstinline|\sphere| \\
  Agnular weight & $\aqweight$ & \lstinline|$\aqweight$| \\
  Sphere integral & $\intsp$ & \lstinline|$\intsp$| \\
  Half sphere integral & $\inthalfsp$ & \lstinline|$\inthalfsp$| \\
  Polar integral & $\intpolar$ & \lstinline|$\intpolar$| \\
  Negative partial polar integral & $\intnpolar$ & \lstinline|$\intnpolar$| \\
  Positive partial polar integral & $\intppolar$ & \lstinline|$\intppolar$| \\
\end{supertabular}

\subsection{Fractions}
\begin{supertabular}{p{0.4\textwidth}|p{0.3\textwidth}|p{0.3\textwidth}}
  Name & Symbol & Command \\ \hline
  Halfs &  $\half$  &  \lstinline|$\half$| \\
   & $\half[3]$  &  \lstinline|$\half[3]$|\\
  Thirds &  $\third$  &  \lstinline|$\third$| \\
   & $\third[2]$  &  \lstinline|$\third[2]$|\\
  Fourth &  $\fourth$  &  \lstinline|$\fourth$| \\
   & $\fourth[3]$  &  \lstinline|$\fourth[3]$|\\
\end{supertabular}

\section{Symbols}
\subsection{FEM}
\begin{supertabular}{p{0.4\textwidth}|p{0.3\textwidth}|p{0.3\textwidth}}
  Name & Symbol & Command \\ \hline
  Domain &  \domain  &  \lstinline|\domain|\\
  Boundary &  \boundary  &  \lstinline|\boundary|\\
  Vacuum Boundary &  \vboundary  &  \lstinline|\rboundary|\\
  Reflective Boundary &  \rboundary  &  \lstinline|\vboundary|\\
  Interface Boundary &  \interface  &  \lstinline|\interface|\\
  Testfunction &  \testfct  &  \lstinline|\testfct|\\
  Angular Testfunction &  \atestfct  &  \lstinline|\atestfct|\\
  Suface Normal &  \normal  &  \lstinline|\normal|\\
  Boundary Normal &  \bnormal  &  \lstinline|\bnormal|\\
\end{supertabular}

\subsubsection{DFEM}
\begin{supertabular}{p{0.4\textwidth}|p{0.3\textwidth}|p{0.3\textwidth}}
  Name & Symbol & Command \\ \hline
  Jump & $\jump{a}$ & \lstinline|$\jump{a}$| \\
  Jump 2 & $\jmpa{a}$ & \lstinline|$\jmpa{a}$| \\ better name here
  Mean value & $\meanval{a}$ & \lstinline|$\meanval{a}$| \\
\end{supertabular}

\subsection{Physics}
\begin{supertabular}{p{0.4\textwidth}|p{0.3\textwidth}|p{0.3\textwidth}}
  Name & Symbol & Command \\ \hline
  Mass flow &  \mdot  &  \lstinline|\mdot| \\
\end{supertabular}

\subsection{Nuclear Symbols}
\begin{supertabular}{p{0.4\textwidth}|p{0.3\textwidth}|p{0.3\textwidth}}
  Name & Symbol & Command \\ \hline
  \sn &  \sn  &  \lstinline|\sn| \\
  & \sn[8]  &  \lstinline|\sn[8]|\\
  \pn & \pn  &  \lstinline|\pn| \\
  & \pn[8]  &  \lstinline|\pn[8]|\\
  Mulitplication factor & \keff  & \lstinline|\keff|\\
  Infintie multiplication factor & \kinf & \lstinline|\kinf|\\
\end{supertabular}

\subsection{Transport}
\begin{supertabular}{p{0.4\textwidth}|p{0.3\textwidth}|p{0.3\textwidth}}
  Name & Symbol & Command \\ \hline
  Streaming direction &  \direction  &  \lstinline|\direction|\\
  Spatial postion &  \position  &  \lstinline|\position|\\
  Current &  \current  &  \lstinline|\current| \\
   & \current[g]  &  \lstinline|\current[g]|\\
  Positive partial current &  \ppcurrent  &  \lstinline|\ppcurrent| \\
   & \ppcurrent[g]  &  \lstinline|\ppcurrent[g]|\\
  Negative partial current &  \npcurrent  &  \lstinline|\npcurrent| \\
   & \npcurrent[g]  &  \lstinline|\npcurrent[g]|\\
  Drift vector &  \drift  &  \lstinline|\drift| \\
   & \drift[g]  &  \lstinline|\drift[g]|\\
  Diffusion coefficient &  \DC  &  \lstinline|\DC| \\
   & \DC[g]  &  \lstinline|\DC[g]|\\
  Nonlocal diffusion tensor &  \DCNL  &  \lstinline|\DCNL| \\
   & \DCNL[g]  &  \lstinline|\DCNL[g]|\\
  Total cross section &  \sigt  &  \lstinline|\sigt| \\
   & \sigt[g]  &  \lstinline|\sigt[g]|\\
  Scattering cross section &  \sigs  &  \lstinline|\sigs| \\
   & \sigs[g]  &  \lstinline|\sigs[g]|\\
  Fission cross section &  \sigf  &  \lstinline|\sigf| \\
   & \sigf[g]  &  \lstinline|\sigf[g]|\\
  Removal cross section &  \sigr  &  \lstinline|\sigr| \\
   & \sigr[g]  &  \lstinline|\sigr[g]|\\
  Absorption cross section &  \siga  &  \lstinline|\siga| \\
   & \siga[g]  &  \lstinline|\siga[g]|\\
  Transport cross section &  \sigtr  &  \lstinline|\sigtr| \\
   & \sigtr[g]  &  \lstinline|\sigtr[g]|\\
  Scattering moments cross section &  \sigl{l}  &  \lstinline|\sigl{l}| \\
   & \sigl[g]{l}  &  \lstinline|\sigl[g]{l}|\\
  Total cross section &  \Sigt  &  \lstinline|\Sigt| \\
   & \Sigt[g]  &  \lstinline|\Sigt[g]|\\
  Scattering cross section &  \Sigs  &  \lstinline|\Sigs| \\
   & \Sigs[g]  &  \lstinline|\Sigs[g]|\\
  Fission cross section &  \Sigf  &  \lstinline|\Sigf| \\
   & \Sigf[g]  &  \lstinline|\Sigf[g]|\\
  Removal cross section &  \Sigr  &  \lstinline|\Sigr| \\
   & \Sigr[g]  &  \lstinline|\Sigr[g]|\\
  Absorption cross section &  \Siga  &  \lstinline|\Siga| \\
   & \Siga[g]  &  \lstinline|\Siga[g]|\\
  Transport cross section &  \Sigtr  &  \lstinline|\Sigtr| \\
   & \Sigtr[g]  &  \lstinline|\Sigtr[g]|\\
  Scattering moments cross section &  \Sigl{l}  &  \lstinline|\Sigl{l}| \\
   & \Sigl[g]{l}  &  \lstinline|\Sigl[g]{l}|\\
  Spatial weight function &  \weight  &  \lstinline|\weight| \\
   & \weight[g]  &  \lstinline|\weight[g]| \\
\end{supertabular}

\section{Units}
\begin{supertabular}{p{0.4\textwidth}|p{0.3\textwidth}|p{0.3\textwidth}}
  Name & Symbol & Command \\ \hline
  Scalar flux & 1\sfluxunit  &  \lstinline|1\sfluxunit|\\
  Anglular flux & 2\afluxunit  &  \lstinline|2\afluxunit|\\
  Diffusion coefficient & 3\dcunit  &  \lstinline|3\dcunit| \\
\end{supertabular}

\end{document}
